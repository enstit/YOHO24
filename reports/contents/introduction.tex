%!TEX TS-program = pdflatex
%!TEX root = ../main.tex
%!TEX encoding = UTF-8 Unicode


\section[Introduction]{Introduction}

	\begin{frame}{Contents}
			
		\tableofcontents
		
		\note{
			\dots			
		}		
		
	\end{frame}
	
	\begin{frame}{Audio Segmentation and Sound Event Detection}
		\dots		
		
		\note{
			\dots
		}
	\end{frame}
	
	\begin{frame}{Datasets}
	
		Common datasets for Audio Segmentation and Sound Event Detection problems are:
		
		\begin{itemize}
			\item \textbf{TUT Sound Event Detection}: primarily consists of street recordings with traffic and other activity, with audio examples of \SI{2.56}{\second} and a total size of approximately \SI{1.5}{\hour}. It has six unique audio classes—Brakes Squeaking, Car, Children, Large Vehicle, People Speaking, and People Walking;
			\item \textbf{Urban-SED}: purely synthetic dataset, with audio example of \SI{10}{\second} and a total size of about \SI{30}{\hour}. It has ten unique audio classes -- Air Conditioner, Car Horn, Children Playing, Dog Bark, Drilling, Engine Idling, Gun Shot, Jackhammer, Siren, and Street Music.
		\end{itemize}
		
		An example of Urban-SED label is:\\
		\centerline{\texttt{[(gun\_shot, 0.3, 1.11), (car\_horn, 0.31, 1.41)]}}
		meaning that an occurrence of gun shot is present from the \SI{0.3}{\second} to \SI{1.11}{\second}, and a car horn from \SI{0.31}{\second} to \SI{1.41}{\second}.
		
		\note{
			\dots
		}
	\end{frame}
	
	\begin{frame}{Metrics}
	
		A popular toolbox for Polyphonic Sound Event Detection models evaluation is \textbf{SED Eval}\footcite{app6060162}.
	
		\begin{multicols}{2}
  			\begin{equation*}
    			\text{Precision} = \frac{\text{TP}}{\text{TP} + \text{FP}}
  			\end{equation*}\break
  			\begin{equation*}
    			\text{Recall} = \frac{\text{TP}}{\text{TP} + \text{FN}}
  			\end{equation*}
		\end{multicols}

		$$
			\text{F$_{1}$ score} = 2 * \frac{\text{Precision} \times \text{Recall}}{\text{Precision} + \text{Recall}}
		$$
		
		\note{
			\dots
		}
	\end{frame}