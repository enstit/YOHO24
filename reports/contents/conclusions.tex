%!TEX TS-program = pdflatex
%!TEX root = ../main.tex
%!TEX encoding = UTF-8 Unicode


\section[Conclusions]{Conclusions}
	
	\begin{frame}{Project workflow}
		\begin{figure}
			\centering
			\begin{tikzpicture}
   				\node[anchor=south west,inner sep=0] at (0,0) {\includegraphics[width=.6\textwidth]{./images/horse.jpg}};
				
				\node[draw,align=center,fill=white,inner sep=1pt,font = {\scriptsize}] at (1.75,.2) {First\\month};
    	
				\draw[thick] (3.25,0) -- (3.25,5.65);

				\node[draw,align=center,fill=white,inner sep=1pt,font = {\scriptsize}] at (3.8,.2) {Second\\month};
    		
				\draw[thick] (4.35,0) -- (4.35,5.65);

				\node[draw,align=center,fill=white,inner sep=1pt,font = {\scriptsize}] at (4.9,.2) {Third\\month};

    			\draw[thick] (5.5,0) -- (5.5,5.65);
			
				\node[draw,align=center,fill=white,inner sep=1pt,font = {\scriptsize}] at (7,.2) {One week\\to the exam};

			\end{tikzpicture}

			\caption{The roadmap of our journey.}
			\label{fig:roadmap}
		\end{figure}
	\end{frame}
	
	\begin{frame}[allowframebreaks]{Lessons learned}
	We learned many things during this project:
	\begin{enumerate}
		\item how to exploit computer vision approaches on different input types (e.g., audio files);
		\item to hyper-tune High Performance Computer resources in a Deep Learning scenario to make the best we can with what we have in our hands;
		\item not all papers found in the literature meet the necessary requirements for replicability. We faced several challenges during the implementation of the algorithm, but scientific researches should be accessible to anyone that has a basic understanding of what is being discussed;
	\end{enumerate}
	
	\framebreak
	
	And, most important, we learned that\dots
		
	\begin{quote}
    	It's all about the journey, not the destination.
    \end{quote}
	Thank you for your attention.
     
	\end{frame}